\documentclass{article}

\usepackage[margin=1in]{geometry}
\usepackage{natbib}

\title{Ideas for this semester's project}
\date{Due Jan 30, 2017}
\author{Rachael Steiner}

\begin{document}

\maketitle

At the beginning of last semester, the ``about my interests'' paragraph
that I submitted talked about my interest in lexical knowledge, its
acquisition, and the implications for the structure of its
relationships. As a shorter-term application to my research, I discussed
my desire to learn how to apply dynamics to the graph-theoretical models
of ``the lexicon.'' My loftier if-I-could-make-any-dynamical-system goal
was to eventually model lexical acquisition with respect to the
relationships among the novel words and the existing words, as well as
the properties of the environment.

With these ideas in mind and now trying to plan a term project, it seems reasonable
to start with the one that can be more readily applied to my research, to provide
a foundation that might be able to be expanded over time into the loftier goal.
Expanding on an existing study that makes functional predictions from a structural
network would be a place to start.

\subsection{Extending findings from graph theory in spoken word
recognition}\label{extending-findings-from-graph-theory-in-spoken-word-recognition}

Graphs of lexical knowledge can be constructed based on relationships among
phonological word forms and used to inform the study of spoken word recognition
\citep{vit08}. There is evidence to suggest that this "whole lexicon" approach
may be more informative than existing models of spoken word recognition which
use only word-word relationships. For example, \citet{ChVit09} constructed a
network where words were connected if they differed by at most a single-phoneme
deletion, addition, or substitution, the same criteria used to determine neighbors
by \citet{lp98}. They predicted that the time needed to recognize a word would
increase as a function of the word's clustering coefficient, or the extent to
which a node's neighbors are also neighbors with one another \citep{ws}.
Importantly, they predicted that this effect would be observed even after
accounting for the
traditional neihgborhood probability rule, which predicts that the time needed
to recognize a word will increase with its number of neighbors or degree \citep{lp98}.

These predictions were supported, which provides preliminary evidence that
graph-theoretical models of lexical knowledge may provide useful information not available
in classical models by including relationships more complex than simple target-neighbor
ones "for free." For example, the authors speculate that there may be some sort
of spreading activation during the process of spoken word recognition, where activation
may spread from one neighbor to another. In this case, words of a higher clustering
coefficient would be harder to recognize because their neighbors would be spreading
activation among one another, increasing competition.

However, as the network was purely structural, the above speculation was not
actually modeled. I would like to try to apply dynamics to the graph to see
if the observed effect could in fact be modeled in terms of spreading activation.
I believe that this will involve random walks, which I will need to learn about,
and I have identified two papers that can provide a starting place \citep{r08, noh2004}.

\bibliography{project-ideas}
\bibliographystyle{newapa}

\end{document}
